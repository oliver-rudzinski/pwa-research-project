\begin{absolutelynopagebreak}
	\begin{abstract}
    	Apps, also Anwendungen für Smartphones, werden über einen Marktplatz des Betriebssystem-Herstellers vertrieben. Die \acf{pwa} bricht mit diesem Konzept und lässt Nutzer eine Webanwendung über den Browser auf dem Gerät installieren. Damit verspricht die \ac{pwa} einige Vorteile gegenüber nativen Apps, z.B. Plattformunabhängigkeit und Verwendung von etablierten Webtechnologien.
    	
    	Diese Arbeit vergleicht die Entwicklung nativer Apps mit derer \ac{pwa}s und evaluiert die Technologien anhand mehrerer Kriterien. Dabei wird auf die technischen Grundlagen eingegangen und exemplarisch eine To-Do-Anwendung für iOS und als \ac{pwa} konzipiert und implementiert, welche den anschließenden Evaluationsprozess stützt.
    	
    	Nach Abschluss der Evaluation lässt sich die Fragestellung nicht klar beantworten. Die Beibehaltung von nativen Apps wird zunächst suggeriert, jedoch wird auch das Potential der \ac{pwa} erkannt.
	\end{abstract}
	
	\selectlanguage{english}
	
	\begin{abstract}
		Mobile apps are distributed via a marketplace managed by the \acsu{os} vendor. The \acf{pwa} disregards this concept by allowing users to install web applications on the device using the browser. The \ac{pwa} promises several advantages compared to native mobile apps, such as platform independence
		and the usage of established web technologies.
		
		This research paper compares the development of native apps and \acp{pwa} and evaluates the
		technologies based on several criteria. The technical basics are discussed and an exemplary to-do application is designed and implemented, both as an iOS app as well as a \ac{pwa}. This procedure supports the subsequent evaluation process.
		
		The evaluation does not yield a definitive answer for the research question at hand. It is suggested to maintain the focus on native apps. Nevertheless, the potential value of \acp{pwa} is recognized.
	\end{abstract}
\end{absolutelynopagebreak}