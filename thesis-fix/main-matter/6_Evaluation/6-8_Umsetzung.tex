\begin{tabbing}
	mmmmmmmmmmmmm				\= \kill
	\textbf{Wertung native App}: \> $++$ \\
	\textbf{Wertung \ac{pwa}}: \> $+$
\end{tabbing}

\subsubsection{Bewertung native App}
In puncto Umsetzbarkeit konnten alle Anforderungen der in Kapitel \ref{chap:architektur} definierten Architektur realisiert werden. Durch die durchgehende Nutzung der durch Apple empfohlenen Architekturstruktur \ac{mvc} war dies auch ohne größeren Aufwand oder die zwingende Nutzung von Umwegen möglich.

Im abgegrenzten Rahmen der Umsetzbarkeit wurden alle Anforderungen erfüllt, weswegen das Kriterium als gut bewertet werden kann.

\subsubsection{Bewertung \ac{pwa}}
Die Kombination aus dem Framework Angular, einer \ac{pwa} und der Hosting-Lösung Firebase funktioniert in der Praxis bemerkenswert reibungslos. Angular reduziert den Implementierungsaufwand gegenüber \textit{vanilla} JavaScript durch saubere Strukturierung der Anwendung in Komponenten und Services. In der Praxis war es nicht möglich, Benachrichtigungen ohne Netzwerkverbindung zu senden. Das ist ein großes Manko im Vergleich zur nativen App. Das Testen der \ac{pwa} war lokal nicht möglich bzw. mit nicht vertretbarem Aufwand realisierbar.

Dieses Kriterium ist mit eher gut zu werten.

