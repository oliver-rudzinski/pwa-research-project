\begin{tabbing}
	mmmmmmmmmmmmm				\= \kill
	\textbf{Wertung native App}: \> $++$ \\
	\textbf{Wertung \ac{pwa}}: \> \Circle
\end{tabbing}

\subsubsection{Bewertung native App}
Folgende Erkenntnisse basieren nicht auf der Entwicklung der Beispielanwendung, können jedoch zu einem ausreichenden Grad nachvollzogen werden (vgl. Abs. \ref{subsubsec:use-ios}), um sie in die Bewertung mit einfließen zu lassen.

 Native Anwendungen werden über einen zentralen Bezugspunkt gefunden, heruntergeladen und installiert. Der Nutzer kann über Stichpunkte und Suchbegriffe nach einer Vielzahl von Apps suchen und die gewünschte Anwendung herunterladen. Homepages, welche ihre Inhalte ebenfalls in Form von nativen Apps bereitstellen, zeigen dies oft in der Mobilversion ihrer Homepage an, sodass eine direkte Verknüpfung erstellt wird. Zwar könnte als Kritikpunkt angesehen werden, dass eine Homepage direkt als \ac{pwa} bezogen werden könnte. Dafür müsste jedoch jede Homepage die Möglichkeit einer \ac{pwa}-Installation bereitstellen, was zurzeit nicht der Fall ist.
 
 Sowohl unter dem Apple App Store als auch im Play Store von Google werden Apps vor ihrer Veröffentlichung geprüft, sodass keine Probleme (vor allem im Hinblick auf die Sicherheit des Endgerätes) bei der Benutzung entstehen. Es besteht in allen Fällen eine Qualitätskontrolle, welche unvorhersehbares Verhalten der bezogenen App vermeidet.
 
 Da sich kein unentkräftbares Argument gegen die Installationskriterien finden lässt, kann dieser Prozess mit gut bewertet werden.
 
\subsubsection{Bewertung \ac{pwa}}
Die \ac{pwa} wird über den Browser installiert. Daher ist die Installation stark abhängig vom verwendeten Browsertyp. Teilweise ist die Installation der Anwendung bedingt durch den Browser überhaupt nicht möglich. Bezogen auf den Desktopbrowser Chrome dürfte es in der Praxis häufig vorkommen, dass Nutzer die Bedienelemente zur Installation im Browser nicht als solche wahrnehmen (s. Abb. \ref{fig:dialog_install_pwa_desktop}). Dies wird als negativ gewertet.

Die Installation ist für den Nutzer nicht nachvollziehbar. Er sieht nicht, dass bzw. wie viele Daten bereits heruntergeladen wurden. Der ein oder andere Nutzer wird sich möglicherweise nicht bewusst sein, dass \textit{Zum Startbildschirm hinzufügen} eine tatsächliche Installation ausführt. Die Installation dauert je nach Komplexität der Webanwendung keine ganze bis wenige Sekunden. Die Deinstallation auf Android Geräten erfolgt wie bei nativen Apps. Bei der Desktop-\ac{pwa} ist diese unter Windows sogar deutlich einfacher, als bei Desktopanwendungen.

Insgesamt ist die Installation einer \ac{pwa} sehr einfach und mit nur einem Befehl des Nutzers ausführbar. Das gilt aber nur dann, wenn der Nutzer das Konzept der \ac{pwa} begriffen hat, was nach heutigem Stand wohl nicht der Fall sein dürfte. Die Installation selbst kommt ohne Lizenzvereinbarungen oder Installationspfade daher und ist außerdem bemerkenswert schnell.

Da einige störende Punkte für den Nutzer nicht optimal sind, die Installation insgesamt aber ein einfacher Prozess ist, erhält dieses Kriterium eine neutrale Wertung.