Anwendungen oder Programme, welche für Smartphones oder Tablets entwickelt wurden, werden gemeinhin als \textit{Apps} bezeichnet. Der Name stammt hierbei aus dem Englischen: eine \textit{App} ist die Kurzform von \textit{Application}. Ins Deutsche könnte man \textit{Application} wörtlich mit Anwendung übersetzen.

Mit der Annäherung von Desktopcomputern über Notebooks an Tablets, welche eine Vielzahl mobiler Geräte wie Touchnotebooks, 2-in-1 Notebooks oder Tablets mit Tastatur und vollwertigem Betriebssystem auf den Markt brachte ist der Begriff \textit{App} nicht mehr scharf definiert.
Längst bezeichnet Windows Anwendungen, die über den Microsoft Store heruntergeladen werden können, als \textit{Apps}. Es ist festzuhalten, dass jedenfalls im deutsches Sprachgebrauch \textit{mobile Apps} meist abgekürzt als \textit{Apps} bezeichnet werden.

Sowohl unter Windows, als auch unter iOS und Android stellt der Betriebssystem-Hersteller Apps den Nutzern über einen Marktplatz zur Verfügung. Für dieses Konzept existieren mehrere Namen, im umgangssprachlichen wird dieser Marktplatz aber meist als \textit{App Store} bezeichnet, obwohl dies eigentlich der Eigenname Apples Anwendungsmarktplatzes ist. 
Ein solcher Marktplatz unterstützt Nutzer dabei, gezielt Apps zu suchen und zu installieren. Der Marktplatzbetreiber wickelt Zahlungen zwischen Nutzer und App-Entwickler ab und bietet Nutzern die Möglichkeit Apps zu bewerten. 
Die \acf{pwa} bricht mit diesem Konzept und lässt Nutzer eine Anwendung über einen Browser installieren. 

Zu jeder App existiert eine Verknüpfung beziehungsweise ein Startschaltfläche auf dem Startbildschirm des Betriebssystems.