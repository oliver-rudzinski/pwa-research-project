\textbf{Wertung \ac{pwa}}: negativ\\
\textbf{Wertung native App}:  \\

\subsection{\ac{pwa}}
Die \ac{pwa} setzt sich aus drei programmiersprachlichen Komponenten zusammen: JavaScript, HTML und CSS. Damit erfordert die Entwicklung sowohl Kenntnisse in prozeduraler Programmierung, als auch in der Implementierung passender HTML und CSS Strukturen, welche letztendlich nur durch JavaScript modifiziert werden.

Außerdem ist zu erwähnen, dass komplexe JavaScript Anwendungen ohne Frameworks und Bibliotheken in der Praxis selten zu finden sind. Die Nutzung von JavaScript ohne beispielsweise Angular, React oder Vue.js ist mit nicht vertretbarem Aufwand verbunden. Da die Nutzung eines Frameworks quasi notwendig ist, aber es zwischen jenen deutliche Unterschiede gibt, zählt dies ganz klar zu den Wissensvoraussetzungen.

Dazu kommen auch zwingend Kenntnisse der Linux-Kommandozeile für node.js, npm und wahrscheinlich eines \ac{cli}-Tools für das Deployment.

Die Wissenshürde ist deutlich erkennbar und für erfahrene Programmierer ohne Webkenntnisse dennoch vorhanden. Dieses Kriterium wird als negativ eingestuft.  