In diesem Kapitel (\ref{chap:einleitung}) wurden die aktuellen Marktenwicklungen kurz mit Zahlen benannt und die Motivation dieser Arbeit dargelegt. Das folgende Kapitel (\ref{chap:grundlagen} \nameref{chap:grundlagen}) legt vorwiegend technischen Grundlagen für die spätere Implementierung einer Anwendung als PWA und nativer App. Dabei wird auf die verwendeten Technologien und Frameworks eingegangen und speziell die PWA auf technischer Ebene erklärt. 

Kapitel \ref{chap:architektur} (\nameref{chap:architektur}) erläutert die Architektur der entwickelten Anwendung: eine Todoliste. In diesem Kapitel werden detaillierte Spezifikationen beschrieben, die den Entwicklungsprozess vergleichbar machen. %Außerdem wird auf architekturbezogene Entscheidungen der Plattformen eingegangen, beispielsweise die Gründe für die Wahl der Frameworks der PWA.

Um den Entwicklungsprozess evaluieren zu können, wird in Kapitel \ref{chap:framework} (\nameref{chap:framework}) der Vergleichsprozess beschrieben und Kriterien mit ihrer Gewichtung aufgestellt und erläutert. 

Anschließend werden im zweigeteilten Kapitel \ref{chap:implementierung} (\nameref{chap:implementierung}) die Implementierungsprozesse der PWA und der nativen App detailliert dokumentiert.

Nach dem Sammeln von Erfahrungen bei der Implementierung werden in Kapitel \ref{chap:evaluation} (\nameref{chap:evaluation}) beide Technologien mithilfe der Kriterien aus Kapitel \ref{chap:framework} verglichen und evaluiert.

Abschließend gibt Kapitel \ref{chap:reflexion} (\nameref{chap:reflexion}) ein Urteil über den Erfolg dieser Arbeit und gibt einen Ausblick auf die Zukunft der PWA.