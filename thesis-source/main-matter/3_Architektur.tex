Die hier zu entwickelnde Anwendung dient zum Anlegen und Verwalten von Aufgaben der Nutzer. Somit löst sie die sogenannte, analoge \textit{To-Do-Liste} ab. Dabei hat die Anwendung (und somit auch die Studienarbeit) keinen Anspruch auf Innovationsdarbietung. Die Begründung der dieses speziellen Entwicklungsbeispiels liegt darin, dass eine To-Do-Listen-Anwendung ein großes Spektrum von Funktionen abbilden kann. Dieses Spektrum reicht von grundlegenden Funktionen (bspw. dem bloßen Anlegen von Aufgaben) bis zu komplexeren Inhalten (bspw. automatischen Push-Notifications über unerledigte oder überfällige Aufgaben). Diese design- und architekturbedingenden Entscheidungen werden im Folgenden definiert und näher beschrieben.

Die Beschreibung der Architektur einer zu entwickelnden Applikation ist eine maßgebende Disziplin im Software-Engineering-Prozess. Dieser Prozess geschieht vor Beginn der Implementierung und ermöglicht, bezogen auf den Umfang dieser Studienarbeit, die Vergleichbarkeit der Applikation hinsichtlich der relevanten Entwicklungsplattformen. Der Umfang sämtlicher Software-Engineering-Prozesse wird grundsätzlich in großen Entwicklerteams praktiziert. Diese gehen der Entwicklung meist komplexer und skalierbarer Anwendungen nach. Im Vergleich dazu ist die hier zu entwickelnde Mobilanwendung lediglich Mittel zum Zweck für die Beantwortung der Forschungsfrage. Das Entwicklungsteam der To-Do-Anwendung besteht aus zwei Personen, welche sich im Rahmen dieser Arbeit autark mit unterschiedlichen Entwicklungsplattformen beschäftigen.

Dies sorgt dafür, dass sich lediglich eine abgespeckte Form des Software-Engineering auf dieses Projekt anwenden lässt. Konkret bedeutet dies, dass keine spezifischen Aussagen über den Software-Prozess bzw. über das Entwicklungsmodell (Wasserfall-Modell, iteratives Modell, etc.) gemacht werden. Dies würde sich hinsichtlich des verhältnismäßig geringen Entwicklungs- und Wartungsaufwands der App kontraproduktiv auf die Zielorientierung auswirken. Umso wichtiger ist die Definition funktionaler sowie nicht-funktionaler Anforderungen, welche daraufhin näher erläutert und spezifiziert werden. Auch die Interaktion zwischen Nutzer und Anwendung muss für eine vergleichende Entwicklung definiert werden. Die zeitlichen und komponentenabhängigen Abläufe innerhalb der App gehören ebenfalls zu den Bestandteilen der Architektur. Letztere beiden Punkte werden im Rahmen des sog. \textit{Systems Modelling} definiert. Darüber hinaus werden auch optische Aspekte und Verhaltensweisen des \ac{ui} abgegrenzt.

\section{Anforderungsdefinition} \label{sec:3-anfoderungen}
\input{main-matter/3_Architektur_Anforderungsdefinition}

\section{Speicherung der Daten} \label{sec:3-speicherung-daten}
Die Datenbank muss auf eine Weise angelegt werden, dass ihre Daten (d.\ h. To-Do-Einträge) einem bestimmten Schema folgen. Konkret bedeutet dies folgende Attribute für die Entität \texttt{ToDo}:

% TODO Tabellarische Darstellung der Attribute
\begin{itemize}
	\item[\texttt{id}] (alpha-)numerische Zeichenfolge, welche einen Eintrag eindeutig erkennbar macht (String)
	\item[\texttt{text}] anzuzeigender Text, welcher den eigentlichen Eintrag darstellt und beschreibt (String)
	\item[\texttt{done}] Status über die Erledigung des entsprechenden Eintrages (Bool'scher Wert)
	\item[\texttt{priority}] Status über die Priorität des entsprechenden Eintrages (Bool'scher Wert)
\end{itemize}

% TODO Beispiel der idealisierten Datenbankeinträge
Unabhängig von der individuellen Architektur der jeweiligen Apps folgt dieses triviale Schema dem Konzept relationaler Datenbanken und könnte somit in einer einfachen Tabelle dargestellt werden.

Um auf die Datenbank zugreifen zu können, muss diese mit entsprechenden Funktionen ausgestattet werden. Neben dem bloßen Erstellen von Einträgen, müssen diese abgegriffen (engl. \textit{fetch}) sowie bearbeitet und gelöscht werden können. Die beiden letztgenanten Funktionen haben bei Ausführung nur Einfluss auf einen durch den Nutzer ausgewählten Eintrag. Somit müssen diese Funktionen das \textit{Objekt} des entsprechenden Eintrages übergeben bekommen. Weiterhin gliedert sich die Bearbeitung von Einträgen in drei Teilfunktionen auf, nämlich dem Ändern des \texttt{done}- oder \texttt{priority}-Attributes sowie dem Ändern des beschreibenden Textes des Eintrags.

Da die Ausführung sowie Umsetzung dieser Operationen mit der \ac{ui} Hand in Hand geht, wird die Definition dieser vorgezogen.


\section{Benutzeroberfläche (\acs{ui})} \label{sec:3-ui}
Die Benutzeroberfläche einer solch einfachen To-Do-Anwendung besteht aus einer Ansicht. Diese Ansicht lässt sich hierarchisch definieren. Die oberste Ebene dieser Hierarchie bildet der hier sog. \textit{App-Container}. Anders als in aufwändigeren Applikationen kann dieser hier mit der Ansicht gleichgesetzt werden, da keine weiteren Ansichten existieren. Dieser Container beherbergt eine Listenansicht, welche für die Darstellung und Interaktion mit den bereits vorhandenen Einträgen zuständig ist. Für das Erstellen der Einträge steht ein separates Text(eingabe)feld zur Verfügung, sowie ein \textit{Button} zur Bestätigung der Eingabe.

Die Listenansicht besteht nun aus mehreren Listeneinträgen (im Folgenden \texttt{Zellen} genannt). Eine Zelle ist für die Darstellung und Interaktion für genau einen To-Do-Eintrag zuständig. Um dies zu ermöglichen, besitzt jede Zelle, neben eines Textfeldes zum Anzeigen des To-Do-Textes, weitere Buttons zum Setzen der Priorität und des Status sowie zum Löschen des Eintrages. Während der Button, welcher für das Entfernen des Eintrages verwendet wird (dargestellt durch ein Kreuz, statischer optischer Natur ist (d.\ h., er ändert nach einem Tippen sein Aussehen nicht), untermalen die Buttons der To-Do-Zustände die gewählten Einstellungen durch ihr Aussehen. Der Button für Priorisierung, welcher durch ein Sternsymbol dargestellt wird, ist bei aktiver Priorisierung gefüllt. Ist dies nicht der Fall, so ist lediglich der Umriss des Symbols zu erkennen. Gleiches gilt für den Button, welcher anzeigt, ob der Eintrag bereits erledigt ist, dargestellt durch ein Häkchen inmitten eines Kreises.

Die beschriebenen, visuellen Eigenschaften lassen sich nun in einem sog. \textit{Wireframe} zusammenfassen, welches gleichzeitig die optische Grundlage der Entwicklung darstellen wird. Dies ist vor allem aufgrund der unterschiedlichen Entwicklungsplattformen von Relevanz, da das Einhalten bestimmter Standards der entsprechenden Plattformen dafür sorgen könnte, dass der letztendliche Vergleich beider Applikationen hohe Differenzen aufweist.

\begin{figure}[h!]
	\includegraphics[scale=0.5]{img/wireframe.png}
	\centering
	\caption{Wireframe}
	\label{fig:wireframe}
\end{figure}

Eine Besonderheit in der Darstellung lässt sich innerhalb der priorisierten Elemente finden. Um diese weiter hervorzuheben, werden diese an den Anfang der Liste gesetzt. Es entstehen somit zwei Teillisten, welche sich jedoch in derselben Listenansicht befinden. Wird nun ein zuvor nicht-priorisierter Eintrag priorisiert, so wechselt dieser seine Position ans Ende der Liste mit den bereits priorisierten Einträgen (bzw. wird oberhalb der ersten nicht-priorisierten Elements platziert). Alle Einträge zwischen der alten und der neuen Position des gerade betrachteten Eintrags werden um eine Listenposition nach unten verschoben. Bei Entfernen der Priorisierung wird das entsprechende Element nun nicht an seine ursprüngliche Position vor der Priorisierung, sondern an den Anfang der nicht-priorisierten Liste verschoben. Entfernt man also bspw. die Priorisierung des letzten Elements in der priorisierten Liste, ändert sich die Reihenfolge nicht. Dieses Verhalten kann vereinfacht in der untenstehenden Abbildung dargestellt werden.

Grundsätzlich werden alle Einträge in der Reihenfolge dargestellt, wie sie angelegt wurden, mit den eben beschriebenen Ausnahmen.


\subsection{Farbliche Darstellung}
Um ebenfalls für farbliche Konsistenz zu sorgen, werden die beschriebenen Elemente auf Basis der folgenden Tabelle in ihrem Erscheinungsbild konfiguriert:

\begin{table}[h!]
	\centering
	\begin{tabular}{ |c|c|c|}
		\hline
		\textbf{Bezeichnung} & \textbf{HEX-Code} & \textbf{Beispiel}\\
		\hline
		
		
		\hline
		\multicolumn{3}{|c|}{\textbf{Allgemeines}}\\
		\hline
		Hintergrund & \texttt{\#F2F2F2} &\cellcolor[HTML]{F2F2F2}\\
		\hline
		Schriftfarbe & \texttt{\#8C8C8C} &\cellcolor[HTML]{8C8C8C}\\
		\hline
		
		
		\hline
		\multicolumn{3}{|c|}{\textbf{Bedienelemente}}\\
		\hline
		Hintergrund für inaktive Bedienelemente & \texttt{\#CECECE} &\cellcolor[HTML]{CECECE}\\
		\hline
		Checkbox (checked) Hintergrund & \texttt{\#1A66FF} &\cellcolor[HTML]{1A66FF}\\
		\hline
		Schriftfarbe der Checkbox (checked) & \texttt{\#FFFFFF} &\cellcolor[HTML]{FFFFFF}\\
		\hline
	\end{tabular}
	\caption{Farbtabelle} \label{tab:farbtabelle}
\end{table}


% Spaß mit Activity-Diagrammen!
\section{Nutzungszyklus} \label{sec:3-nutzungszyklus}
\input{main-matter/3_Architektur_Nutzungszyklus}

%\end{figure}