\textbf{Wertung \ac{pwa}}: $+$\\
\textbf{Wertung native App}:  \\

\subsection{\ac{pwa}}
Die \ac{pwa} hat keinen Zugriff auf das Dateisystem. Daten werden über den Browser im Speicher abgelegt, beispielsweise im Key-Value-Store \texttt{local storage}. Konfigurationen und Daten können auf diese Weise einfach gespeichert werden.

Das Speichern von binären Daten, wie Bildern, gestaltet sich in der Praxis als schwierig. Es existieren uneinheitliche browserspezifische Lösungen. Höchstwahrscheinlich kommt der Entwickler aber nicht um das Speichern binärer Daten als kodierter Text, wie beispielsweise Base64.

Das Kriterium wird als gut bewertet, weil mit dem Browserspeicher ein Großteil der Anwendungsfälle für Datenspeicherung abgedeckt ist und diese sehr einfach von Entwickler genutzt werden können.